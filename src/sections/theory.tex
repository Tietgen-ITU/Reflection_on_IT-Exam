\section{Theory}
This section is going to explain key theory that is going to be used to answer the questions presented in the introduction. The theories are going to be used in the discussion section. First theory on privacy thereafter ethics is being covered. 

\subsection{Privacy}
% TODO: Highlight and acknowledge that there are more views on concept of privacy and that we only scratch the surface on some of them.
In the philosophical realm there is no common definition of privacy. 
The first well written essay about privacy was introduced by Samuel Warren and Louis Brandeis with the title \textit{The Right to Privacy}. Here they stated that the individual should have the "Right to be alone". Warren and Brandeis examined if existing law is able to protect the privacy of the individual, where they would later explain how and to the extent that the law would protect the individual. 

They focused in a large part on the press and publicity due to the newly inventions of photography and newspapers that could violate and invade privacy by public dissemination. During their essaay they felt that some of the cases that they reviewed could be protected in a more general "right to privacy". This "right to privacy" should protect individuals to the extent that one's thougths, sentiments, and emotions could be shared to others. After the publication of their essay, the public, state and courts were expanding the right to privacy. In order to systematize and describe the rights of privacy, William Prosser presented 4 four different interests in privacy:
\begin{itemize}
    \item[1.] Intrusion upon a person's seclusion or solitude, or into his private affairs
    \item[2.] Public disclosure of embarrassing private facts about an individual 
    \item[3.] Publicity placing one in a false light in the public eye
    \item[4.] Appropriation of one's likeness for the advantage of another \temp{(Prosser 1960, 389)} 
\end{itemize}

Warren and Brandeis laid the foundation of a concept of privacy known as control of information about oneself. Others more recent views of privacy focusing on control over information has also been presented.

% Talk about Alan Westins view on privacy
One of them is Alan Westin which describes privacy as the individuals' ability to determine when, how, and to what extent information about us is communicated \temp{(Westing, 1967)}. 
% Talk about Parents view on privacy
However, William Parent views privacy as the condition of not having personal information known or possessed by others. Parent states that the condition on privacy is as a moral value for thoese for values individuality and freedom. It is not a moral or legal right to privacy. In his perspective personal information is factual. They are facts that we as individuals don't want the public to know about us. This could be information about our health, salary, democratic orientation, etc. The line is drawn when this information is documented for the public i.e. it is in newspapers or other public documents. This means that there is no invasion of privacy once it is public knowledge. Privacy is only violated when new undocumented information is gained about an individual. 

% TODO: Should we talk about the critique of this perspective

% Talk about another perspective of privacy
Another view is privacy in terms of access. Sissela Bok \temp{(1982)} describes this view where privacy protects the access to information about an individual and that we as an individual has exclusive rights to our information \temp{(Look at this again)}. Ruth Garvison agrees by arguing that privacy is related with the concern over accessibility to others. In her perspective privacy is gained by three independent ways that is interrelated, namely through secrecy, anonymity, and solitude. Garvision further defines the three properties of privacy, by stating that secrecy is when no one has information about the individual, anonymity is when no one pays attention to you, and solitude is when no one has physical access to one.  
In other words privacy is about having limited access to the individual. To achieve perfect privacy is when the information is completely inaccessible to others i.e. not violating secrecy, anonymity, and solitude. 

\subsection{Ethics}
Philosophy has three main view on ethics, namely ethics on duty(also known as Deontological ethics), Virtue ethics, Utilitarinism. For that reason this section is going to cover some central theory about these perspectives.

% Duty ethics / denotological ethics
Deontological ethics views a choice or act as ethical by what is moral. Stating that even if the act is considered good then it is not ethical if it isn't moral. By considering if something is ethical or not, the consequences of the choice cannot be taken into considertion. This is due to choice could have good consequences in the future, however it is still morally forbidden to do so. In other words, what is right comes over what is good. 
Kant provided a way to determine if something is ethical or not in the perspective of Deontology. He has created a way to determine whether something is morally permissible or not. It is determined by creating a thought experiment were we create a world with a universal law. The question to answer is not whether it is good if everyone acted like the law was stated, but rather if it is possible will it as a law. If this world creates a paradox then it is not moral. An example could be a world where everyone is allowed to cheat. This creates a world where cheating is not really cheating anymore. For that reason it is not moral and in terms of Deontological ethics it is wrong.

% Utlitarianism
Utlitarianism is a part of the ethics that looks if the choice or action is any good i.e. utilitarians view that the morally right action is the action that does the most good. Utilitarianism wants to maximise utility. Utility is described by Bentham as, a property in any object that creates benefit, advantage, pleasure, good. It can also be considered to be preventing pain, evil or unhappiness for those involved. As an example if you point the gun at individual who is going millions, and you have to decide if you want to kill this man or not. If you kill him, you save millions of people. On the contrary you have to live with killing a person. Killing that person would be an ethical choice in the perspective of utilitarianism as the maximum overall good is to save those people. 

% Virtue ethics
Virtue ethics on the contrary looks at a person's virtues. A virtue is characterized as an excellent trait of characther. This is something that a person really posses'. Such traits is to notice, expect, value, feel, choose, or act in certain ways. In order to say that an individual possess a virtue, the intensions of that individual has to be wholehearted from specific thoughts of reason for doing that action. More specifically if the individual is kind to someone, it is because he recognizes that "otherwise would be mean". This is in contrast to a person who is kind due to the fear of being disliked. The reasoning of taking the action is what makes the person possessing a virtue. 
In virtue ethics not only the virtue but also the motives needs to be considered. Micheal Slote\temp{(Ref)} defines virtue ethics as understanding "rightness in terms of good motivations and wrongness in terms of the having of bad motives"\temp{(ref)}. Also, Zagzebski has defined that a wrong act is something where we would feel quilty doing it as it goes against our virtue and therefore might not do it anyway. 
An example as something that virtue ethics would view as ethical, is that a person is helping a blind person over the road because that person virtues and has the motivation to help. In the eye of Zagzebski, this would be rightfull as the act does go against the that persons virtue and doesn't make him fell bad. However, if the person did it to impress somebody, let us say a girl that he likes, then motivations becomes to impress that girl, and uses the blind man to do so. This would be unethical. 
