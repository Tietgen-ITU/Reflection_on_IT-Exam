\section{Introduction}
Denmark has always been a country that is proud to say that they are an example for the rest of the world when it comes to the green initiatives and saving the climate. There have been a lot of different investments over the years like lowering taxes of electric cars, providing money to get rid of oil-fired boilers, and building huge wind turbine farms in the middle of the sea. However, there has also in the last decade been a push from the governments to limit the amount of cars being driven in big cities like Copenhagen. In the late 2011 and the early 2012 a proposal to let car owners pay for entering the city of Copenhagen with their vehicle was presented. The idea was to force car owners to take public transport when traveling into Copenhagen rather than the car. This obviously infuriated car owners driving to work every day, now having to pay more in taxes, but also arguing that it would take longer for them to get to work because public transport is not fit for the task. 10 years later, the discussion has returned. A climate council, initated by the danish government, has proposed the idea of roadpricing. Roadpricing is a tax that is being added to the driver when the car has been driven around the country \cite{borsen:2013}. The idea is not fully thought out, but the government is currently testing solutions to see how it works in practice \cite{kosmopol:2022}. The key difference is that this is not a one time payment only in Copenhagen. This is a tax for the driver depending on which road the car has been on. While there were heavy disagreement between politicians the last time, now they are more open-minded to the idea. The politicians are mainly looking at this from a financial perspective, concluding that it is a fair solution. However, politicians are not talking about the technology that the government is testing in order to make this happen in practice. The idea is to add a black box to every car. The black box is a GPS tracker that sends information to the danish tax payment institute called SKAT regarding the whereabouts of the driver, including time, date and location \cite{jylland:2011}.
This essay is going to cover the privacy and ethical aspects of the solution, in order to contribute with something broader than just economical persoectives. Specifically, I will try to answer the following research questions:
\begin{itemize}
    \item Does roadpricing with GPS tracking violate the privacy of the common people?
    \item It is ethical to use roadpricing using GPS tracking technology in order to add taxes?
\end{itemize}
I will do so by (1) give an introduction to the problem (2) provide some detail into the history of the GPS technology, how it works, and its use cases, (3) summarize the theory on privacy and ethics, (4) go into a discussion covering the two research quesions above, and (5) deliver a conclusion based on the discussion with respect to the questions answered.