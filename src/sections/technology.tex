\section{GPS - Global Positioning System}
GPS is short for Global Positioning System. It was invented by the US military with a prototype in the 1980s and then later fully operational with 24 satellites in the year of 1993. The first GPS system is called NAVSTAR GPS. Originally, the system was not intended for civilian use, but for the US military to gain an advantage on the battlefield. However, after an incident of accidentally shooting a plane down, the United States of America issued a bill opening the tracking technology to the public. This service was called the GPS Standard Positioning Service(GPS SPS). However, GPS SPS muddied the signal, thus making the positioning imprecise. In fact, it was having a 100-meter precision whereas today we have a precision down to the centimeter. The NAVSTAR GPS system is still maintained by the USA \cite{wiki:gps}. This has caused other countries, like Russia and China, to implement their own GPS system.

The NAVSTAR GPS was implemented by having three components, namely satellites, ground stations, and receivers. Receivers are small devices, such as a mobile phone, tablet, computer or even smartwatches. As mentioned, the system has 24 satellites which travel in orbit. The system knows exactly where each satellite is supposed to be at any given moment. It uses the ground station to calculate its distance to four or more satellites that is supposed to be closest to the station. It calculates the distances to the satellites, to be sure that it knows exact position of the satellites. When the distance has been calculated, the exact position of the receiver is calculated and known \cite{nasa:gps}. Later, GPSs have been further developed, thus becoming more precise but also more complex. Some GPS systems launched by different countries now collaborate in delivering precise location. This also makes the systems more resilient to satellite crashes. 

Purpose of a GPS is many. Some of them is used to locating who is nearby in a dating app, to sharing the route that you have taken while jogging. Even some insurance companies use GPS locations to provide a cheaper travel insurance by only providing insurance in the countries that you have visited. However, the downsides are also that this can be leveraged by hackers or other parties to track the location of specific individuals. Companies like Google, Apple and Microsoft are known to use a user's tracking information to provide services and suggestions in real time. Articles about hackers getting this information or selling to third-party companies that do not have control of this data have surfaced at a higher rate. A recent example being the Cambridge Analytica scandal \cite{newyorktimes:facebook}. Moreover, governments are also looking into how GPS technologies can be used and leveraged for different scenarios. A recent example is the roadpricing that the Danish government has proposed. Even the Chinese government has implemented GPS trackers in cars a specific region of the country due to a spike of murders \cite{guardian:2017}.

\subsection{Roadpricing}
As mentioned, roadpricing is the concept of putting taxes on the driver of the car depending on where the car has been and at what time. The Danish government has invested 20 million Danish kroner into reasearching solutions for roadpricing. The technology that keeps getting attention in the media and from the politicians is a small black box getting installed into cars. It primarily consists of a GPS reciever that is activated when the car is turned on. During the trip, the black box tracks the postion of the car and timestamp. The idea is to send the information to the Danish tax authorities, SKAT, for them to perform the calculations of how much the driver should pay in taxes. Since this is in the early stages, and the research of how such a solution could work was initiated early 2023 \cite{kbh-liv:2023}, there is little to no information on precisely what information is being sent to SKAT.