\section{GPS}
GPS is short for Global Positioning System. It was made by the US military with a prototype in the 1980s and then later fully operational with 24 satellites in the year of 1993. This GPS technology is also called NASTAR GPS. Originally the system was not developed for civilian use, but for the US military to gain an advantage. However, after an incident of accidentally shooting a plane down, the United States of America(from now on referred as USA)issued a bill opening the tracking technology to the public. This service was called the GPS Standard Positioning Service. GPS SPS came with scrambling making the positioning imprecise. In fact, it was having a 100-meter precision whereas today we have a precision down to the centimeter. The system is still being maintained by the USA. This has coursed other Countries to implement their own GPS system, like Russia and China.

The NASTAR GPS is implemented by having three components namely satellites, ground stations, and receivers. As mentioned the system has 24 satellites which travels in orbit. We know where they are or are supposed to be at any given moment. It uses the ground station to calculates its distance to the four or more satellites. This is also to make sure that we know the position of the satellites. When the distance have been calculated, the exact position of the receiver is calculated and known. Now GPSs have been further developed to become more precise. And by becoming more precise also more complex. Some GPS systems launched by other countries now collaborates to deliver precise location and to be more resilient to satellites crashes.
Receivers are small devices, such as a mobile phone, tablet, computer or even smartwatches. 

GPS is now used for various purposes from locating who you are nearby in a dating app to sharing the route that you have been out and jogging. Even some insurance companies use GPS locations to provide a cheaper travel insurance by only providing insurance in the countries that you have visited. However, the downsides is also that this can be leveraged by hackers or other parties to track location of specific individuals. Companies like Google, Apple and Microsoft are known to use a users tracking information to provide services and suggestions in real time. Articles about hackers getting this information or selling to third-party companies that does not have control of this data \temp{(Add ref about the Facebook scandal)} have surfaced \temp{more and more}. Not only that but governments are also looking into how GPS technologies can be used and leverage for different scenarios. A recent example of that is the roadpricing that the danish government has proposed. The chineese government has already implemented this for a specific region of the country due to a spike of murders \cite{guardian:2017}.