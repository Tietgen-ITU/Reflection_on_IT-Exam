\section{Discussion}
This section is first going to discuss if the Roadpricing technology is going to violate privacy for the common people. Secondly, the discussion section is going to touch upon some ethical views of using this technology in order to add taxes.
Since the Roadpricing technology is not yet in the final stages, we are going to assume that the technology sends GPS locations and timestamps to the danish tax authorities. Furthermore, that if this becomes a law, that the danish people do not have a choice other than to install this black box GPS device into their cars. 

Looking at the solution from the perspective of Warren and Brandeis right to privacy definition, the danish people should be protected from exposing ones thougts, sentiments and emotions from being shared. As information about locations and timestamp doesn't reveal any information about an individuals intentions thougts or anything else in that matter, it does not violate their definition of privacy. 
% Prosser

However, the four interests of privacy by Prosser, wouldn't agree. 
As his first interests states, the "Intrusion upon a person's seclusion or solitude, or into his private affairs". In Prossers eyes this would be violated, mainly due to the last part of the sentence, "private affairs". By exposing the location of an individual would intrude an individuals private affairs. Private affairs aren't very narrow. If the individual is known for being healthy, but goes out to get an ice cream. Then the location could reveal that the individual has been to a place known for getting an ice cream, thus revealling a private affair to the public.  
Looking at the second interest, it states that "public disclosure of embarrasing private facts about an individual" would violate privacy. At first glance, looking at the locations of an individual wouldn't reveal much information. However, in the situation where the location ends at a location such as red-light district, would for some individuals be an embarrasing private thing to share with the public. Thus, this scenario does reveal embarrasing facts about the whereabouts of an individual, again violating privacy concerns. Prossers third privacy interest, "Publicity placing one in a false light in the public eye" could in some sense be violated. In a world where everyone only drives a car that each individual have bought, wouldn't reveal any false information to the public. In fact, the system is designed to know exactly where the car has been. However, a car could be used by anyone. Since the car is registered to a specific person it would look like the individual owning the car, drove to some particular place in the eye of the public. Since everyone can drive the car, then it could also have been a family member who used the car in the particular moment. This could potentially place a specific person at a location that the individual may not have been at, making it violate the third interest as well. 
The last interest of privacy could also be violated. It states that the "Appropriation of one's likeness for the advantage of another". However, this is only violated in the situation where the information about the whereabouts of the individual is leaked and used in some sort of smear campaign, that puts someone else in a better light.  
% Alan Westin

Alan Westin would agree that Roadpricing violates one's privacy. But does so from another perspective. His view is focused on the control of information i.e. when, how and to what extent the information is used. As this is going to be a law, the issue is that the individual does not have much to say. Meaning, the individual is not free to choose the car as transport without being tracked. Also, the solution about how they are being tracked is not up to the individual either, being that this is imposed by the government. Furthermore, to what extent is not yet stated by government. One could imagine a world where the government initially only shares this information with the danish tax authorities, but later allows the police to use similar information to fight crime. The individual does not have much to say about the control of to what extent their information is used, and for that reason Roadpricing violates Alan Westins view as well.  

% Ruth Gavison
Looking from the perspective of access, Ruth Gavison has a looser view on privacy. According to Garvison there is privacy and perfect privacy. Based on the three properties, namely secrecy, anonymity, and solitude nearly all of them is violated, hence not providing perfect privacy. This is due to the property of solitude, which Garvison defines as "no one has physical access to one". Since Roadpricing only focuses on sending pieces of location information from the individual car, then there is no physical aspect, when it comes to getting access to the individual. However, this does mean that there is some level of privacy since the property, solitude, isn't violated. \\

% TODO: Describe that it violates the secrecy and anonymity properties

%%%%%%%%%%%%%%%%%%%%
% Ethical question %
%%%%%%%%%%%%%%%%%%%%

% deontological
\noindent In order to look at Roadpricing from an ethical perspective, the essay will assess each ethical view, starting with deontological. The Deontological perspective states that consequences of the act cannot be taken into consideration. To assess the act, only the moral reason needs to be considered to be ethical. Looking at Roadpricing, the council proposing it, reasons that making it more expensive to use the car on certain roads, will result in fewer cars being driven. The reasoning behind it, is to reduce $CO_2$ being emitted by cars. Since we do not look at the outcome of the situation, then this reasoning cannot be taken into consideration. However, using the discussion on privacy, it can be argued that the action of tracking and potentially violating one's privacy when driving, isn't moral. Thus, according to deontological ethics, it would not be ethical to implement Roadpricing with GPS tracking, even though the outcome would be good.

% Virtue ethics
From the perspective of virtue ethics, the theory states that the action or choice that an individual makes should make them a better person. There are two views that can be looked at. If the council and government, implmenting Roadpricing acts virtous and therefor ethical, or if the government helps people becoming virtous. The council that is proposing Roadpricing does it with the intention to limit the amount of cars to save the climate. The intention of making the world more clean by limiting the amount of cars used is a good thing to do. However, it is difficult to determine if this is the real intention or not. One could argue that it is also for the state to make more money, and the environmental argument is a cover-up. If the real intention is to save the environment, wouldn't it be better to get rid of diesel cars and make the switch to electric cars? It is hard to say what the real reason is for the government to implement Roadpricing. If the real reason is to save the environment and limit the amount of $CO_2$ emitted from cars then it is virtous for the right reason, making it ethical. However, if the real reason is to make more money, then it is making it unethical.   
Looking at it from the other perspective, if this law is being imposed, could it help the individual to make a choice that makes the individual a better person? The individual would make a good choice by not using the car. However, do they do it for the right reason. This raises the question, if an individual is choosing not to use the car, due to the climate, why do have a car to begin with?  
In fact, some of those who previously choose the car, would propably choose not to, due to economic reasons or that they feel like being under surveillance. This means that even though, it would be virtous not using the car due to the environment, then most of the individuals doing so, would propably do it for another reason. Since the individual isn't wholehearted in the reason making a virtous choise, it would make the act unethical in the perspective of virtue ethics.

% Utilitarian
With the utilitarian perspective we want to maximize the good outcome. There are several arguments in favor of Roadpricing but also against. By impossing Roadpricing even though the government are tracking each individual position, the money that is being gathered can be used to invest in better roads. There is also the possibility that fewer would use their car and instead use public transport limiting the overall $CO_2$ being emitted by cars. At the same time, if the police is allowed access to the location information gained, it could help solve some cases. On the contrary the people might start feeling watched. There is a danger of people feeling less free, when they go for a drive. Misuse of the information is also a possibility