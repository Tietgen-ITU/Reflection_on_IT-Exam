\section{Conclusion}
This eassy has described the current roadpricing situation in Denmark, GPS technology, and theory on privacy and ethics. Furthermore, the discussion used the theory to answer both the privacy and ethical research question. Looking at the privacy discussion, the definitions regarding both control of information and access to information was used. In terms of privacy it shows that Warren and Brandeis definiton has an emotional perspective. As can be seen in the discussion it is also the only definition that says roadpricing isn't violating privacy. 
The argument being that GPS tracking doesn't expose the individuals thougths, sentiments or emotions. On the contrary, Prosser's interests show that all four interests are violated. It shows that some situations could provoke the violation of privacy, especially the disclosure of private affairs. Shifting to privacy regarding control of information, Westin's definition agrees that roadpricing violates privacy. As shown in the discussion, the issue with roadpricing is that the individual can't use the car wihtout being tracked. The individual does not have control on when, how and to what extent they are sharing information. Furthermore, looking at the discussion on control of access, it shows that the roadpricing solution does provide some privacy but not perfect privacy. This is due to roadpricing having to deliver information about who and where, which violates the secrecy and anonymity property. 

Shifting the focus to ethics, in the eyes of Deontological ethics, roadpricing isn't ethical. This is mainly due to the violation of privacy for the majority of definitions regarding the privacy discussion. Since this perspective doesn't look at the outcome of the action, implementing roadpricing and violating privacy is unethical. Virtue ethics agree to some degree. Here the discussion visits two views, the government taking the decision to implement roadpricing and the government implementing roadpricing in order to help people become better persons. However, as shown in the discussion there can be asked questions about the reason, for both views making it a bit unclear if it is ethical or not.

From the Utilitarian perspective, which looks at maximizing the good outcome there are many arguments in favor of roadpricing. One of the arguments being, that roadpricing might reduce emitted $CO_2$ due to having fewer cars on the road. The downside is that the individual might feel that they are under surveillance while driving and they people's privacy might be violated. 

To sum up, a majority of the definitions are saying that roadpricing with GPS violates privacy. The problems revolve around access to the individual and lack of control of information about oneself. Furthermore, the ethical perspectives covered are leaning towards roadpricing being unethical, but this is not completely clear.