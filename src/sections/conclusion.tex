\section{Conclusion}
This eassy has described the situation, technology, and the theory. Furthermore, the discussion used the theory to answer both the privacy and ethical question. Looking at the privacy discussion the definitions regarding both control of information and access to information was used. In terms of privacy it shows that Warren and Brandeis definiton has an emotional perspective. As can be seen in the discussion it is also the only definition that says Roadpricing isn't violating privacy. 
The argument being that GPS tracking doesn't expose the individuals thougths, sentiments or emotions. However, Prossers interests shows that all four interests, gets violated. It shows that some situations could provoke the violation of privacy, especially the disclosure of private affairs. Shifting to privacy regarding control of information, Westins definition agrees that Roadpricing violates privacy. As shown in the discussion, the issue with Roadpricing is that the individual cannot use the car wihtout being tracked. The individual does not have control on when, how and to what extent that they are sharing information. Furthermore, looking at the discussion about control of access, it shows that the Roadpricing solution does provide privacy but not perfect privacy. This is due to Roadpricing having to deliver information about who and where, which violates the secrecy and anonymity property. 

Shifting the focus to ethics, in the eyes of Deontological ethics, Roadpricing isn't ethical. This is mainly due to the violation of privacy for the majority of  definitions regarding the privacy discussion. Since this perspective doesn't look at the outcome of the action, implementing Roadpricing would make it unethical. Virtue ethics agree to some degree. Here the discussion visits two views, the government taken the decision to implement Roadpricing and the government implementing Roadpricing to help people become better persons. However, as shown in the discussion there can be asked questions about the reason, for both views making it a bit unclear if it is ethical or not.

Whereas from the Utilitarian perspective, which does look at maximizing the good outcome has many arguments in favor of Roadpricing. One of the arguments being, that Roadpricing reduces emitted $CO_2$ due to having fewer cars on the road. The downside is that the individual might feel that they are under surveillance while driving. 

Thus concluding the privacy and ethical concerns that, there are a majority of the definitions saying that Roadpricing violates privacy. In fact the defintions are focused on access and control of information. Furthermore, the ethical perspectives covered is leaning towards Roadpricing being unethical.