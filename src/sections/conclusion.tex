\section{Conclusion}
This eassy has described the situation, the technology, and the theory. Furthermore, the discussion used some theory to answer both the privacy and ethical question we have tried to find an answer to the two questions in the introduction regarding privacy and ethics. Looking at the privacy discussion the definitions regarding both control of information and access to information. In terms of privacy it shows that Warren and Brandeis definiton has an emotional perspective. As shown in the discussion according to their view GPS tracking does not violate privacy as it doesn't expose the individuals thougths, sentiments or emotions. However, Prossers interests shows that three out of 4 of them gets violated. It shows that some situations could provoke the violations of privacy, especially the disclosure of private affairs. The defintion of Westin agrees that roadpricing violates privacy. As shown in the discussion, the issue with roadpricing is that the individual cannot use the car wihtout being tracked. The individual does not have control on when, how and to what extent that they are sharing information. Furthermore, looking at the discussion about control of access, it shows that the roadpricing solution does provide privacy but not perfect privacy. This is due to roadpricing having to deliver information about who and where, which violates the secrecy and anonymity property. 

Shifting the focus to ethics, in the eyes of deotological roadpricing isn't ethical. This is mainly due to the violation of privacy for the majority of the definitions regarding privacy. Since this perspective doesn't look at the outcome of the action, implementing roadpricing would make it unethical. Virtue ethics agree to some degree. Here the discussion visits two views, the government taken the decision to implement roadpricing and the government implementing roadpricing to help people become better. However, as shown in the discussion there can be asked questions about the reason, for both views making it a bit unclear if it is ethical or not.

Whereas from the utilitarianism which does look at the perspective of maximizing the good outcome. There is a lot of arguments in favor of roadpricing namely reducing emitted $CO_2$ due to fewer cars on the road, more money to invest in maintaining the quality of roads, and a possibility for the police to gain another tool to fight crime. The downside is that the individual might feel that they are under surveillance while driving. 

Thus concluding the privacy and ethical concerns that, there are a majority of the definitions saying that roadpricing violates privacy. In fact the defintions are focused on access and control of information. However, the ethical perspectives cannot determine whether roadpricing is ethical.